\documentclass[11pt,a4paper,sans]{moderncv}

% moderncv themes
\moderncvtheme[grey]{casual}                 % optional argument are 'blue' (default), 'orange', 'red', 'green', 'grey' and 'roman' (for roman fonts, instead of sans serif fonts)
%\moderncvtheme[green]{casual}                % idem

% character encoding
\usepackage[english]{babel}
\usepackage[utf8]{inputenc}                   % replace by the encoding you are using

\usepackage{lmodern}
%\renewcommand{\sfdefault}{\rmdefault}
% adjust the page margins
\usepackage[scale=0.75]{geometry}
\usepackage{graphicx}
%\setlength{\hintscolumnwidth}{3cm}						% if you want to change the width of the column with the dates
%\AtBeginDocument{\setlength{\maketitlenamewidth}{6cm}}  % only for the classic theme, if you want to change the width of your name placeholder (to leave more space for your address details
%\AtBeginDocument{\recomputelengths}                     % required when changes are made to page layout lengths

%----------------------------------------------------------------------------------
%            Kontaktdaten
%----------------------------------------------------------------------------------
% VORNAME
\firstname{lukas}
% NACHNAME
\familyname{elsner}
%TITEL (optional, ggf. einfach die Zeile löschen!)
\title{curriculum vitae}
%ADRESSE  (optional, ggf. einfach die Zeile löschen!)
%\address{dorfstrasse 11}{25485 langeln}{germany}
%HANDYNUMMER  (optional, ggf. einfach die Zeile löschen!)
%\mobile{0176 20 21 52 67}
%\mobile{(+49) 176 20 21 52 67}
%EMAIL-ADRESSE  (optional, ggf. einfach die Zeile löschen!)
%\email{cv@lukaselsner.de}
%HOMEPAGE  (optional, ggf. einfach die Zeile löschen!)
\homepage{www.lukaselsner.de}
%FOTO  (optional, ggf. einfach die Zeile löschen!)
%  64pt = Höhe des Bildes, 'picture' = Name des Bildes
\photo[96pt]{avatar}

% to show numerical labels in the bibliography; only useful if you make citations in your resume
\makeatletter
\renewcommand{\bibliographyitemlabel}{\@biblabel{\arabic{enumiv}}}
\makeatother

%----------------------------------------------------------------------------------
%            Inhalt
%----------------------------------------------------------------------------------

\begin{document}
\maketitle
\section{}
\cvline{born}{01.11.1984 in Henstedt-Ulzburg, Germany}

\section{work experience}

\cventry{since 2016}{Professional Application Engineer / Build Manager}{SBB CFF FFS}{}{}{
  \begin{itemize}
    \item creating an fully automated iOS/Android build pipeline
    \item developing the next generation continuous integration
    \item application operations manager continuous integration
    \item migration of infrastucture to amazon aws cloud
    \item planning team software engineering security
    \item planning team software quality
    \item software development self service application in spring boot
  \end{itemize}
}

\cventry{since 2004}{Software Engineer / IT-Consultant}{mindrunner}{}{}{
  \begin{itemize}
    \item conducting freelance work as a software engineer and consultant
  \end{itemize}
}
\cventry{2010--2011}{Software Engineer}{freifalt GmbH}{Passau/Hamburg}{}{
  \begin{itemize}
    \item design and implementation of iOS-Apps and their backends
    \item responsible for the company's internal infrastructure, a virtualized, heterogeneous network
  \end{itemize}
}
\cventry{2005--2009}{Software Engineer}{MSA Auer GmbH}{Halstenbek/Berlin}{}{
  \begin{itemize}
    \item developing several core products of the company
  \end{itemize}
}

\section{certification / training}
\cventry{2018}{Certification}{embarc}{}{}{
  \begin{itemize}
      \item Agile Software Architecture
  \end{itemize}
}
\cventry{2018}{Certification}{DasScrumTeam}{}{}{
  \begin{itemize}
      \item Scrum Master
  \end{itemize}
}
\cventry{2018}{Training}{SBB CFF FFS}{}{}{
  \begin{itemize}
      \item Railroad Academy - Train Know-How
  \end{itemize}
}
\cventry{2017}{Certification}{CloudBees}{}{}{
  \begin{itemize}
      \item Certified CloudBees Jenkins Engineer (CCJE)
  \end{itemize}
}

\section{education}
\cventry{2018--2019}{Nanodegree}{Udacity}{}{}{
  \begin{itemize}
      \item Blockchain Developer
  \end{itemize}
}

\cventry{2014--2015}{Master of Information Technology}{Queensland University of Technology}{Brisbane}{Australia}{
  \begin{itemize}
      \item Graded: 7 (High Distinction)
    \item Software Architecture (Advanced)
  \end{itemize}
}
\cventry{2015}{Master Thesis}{\textbf{[Pyhton, C, Java, VHDL, git, \LaTeX]}}{QUT, Schatz Forensic}{}{
  \begin{itemize}
    \item In this project, a low level framework with the goal of reading raw
    NAND flash devices found on SD-Cards and USB-drives was created. The
    hardware/software co-design runs on a Pipistrello FPGA
    board and enables dynamic pin assignment, high speed block
    orientated I/O and voltage variation to particular pins on the
    hardware side, as well as a serial interface to an attached remote machine
    for configuring the hardware and read data from the NAND flash on the
    software side.
  \end{itemize}
}
\cventry{2013--2014}{M.Sc. Information Technology}{University}{Lübeck}{Master of Science}{
  \begin{itemize}
    \item Software Systems Engineering
  \end{itemize}
}
\cventry{2013}{Software Systems Engineering Laboratory}{\textbf{[Java, Maven, Jenkins, git, \LaTeX]}}{Uni Lübeck}{}{
  \begin{itemize}
    \item required course for every Software Systems Engineering student
  \end{itemize}
}
\cventry{2009--2012}{B.Sc. Information Technology}{University}{Passau}{Bachelor of Science}{
  \begin{itemize}
    \item Graded 1.9
    \item Minor: Business Administration
  \end{itemize}
}
\cventry{2012}{Bachelor Thesis}{\textbf{[C++, \LaTeX]}}{Uni Passau}{\url{http://www.uni-passau.de}}{
  \begin{itemize}
    \item planning and implementing an easy-to-use class library on top of a proprietary SDK for prototyping perception platforms with the goal to automate testing within the whole research project (Graded: 1.3)
  \end{itemize}
}
\cventry{2012}{Research Assistant}{\textbf{[C++]}}{Uni Passau}{}{
  \begin{itemize}
    \item developing and troubleshooting of plugins for a perception platform
    \item developing of algorithms for automated image and pattern recognition
  \end{itemize}
}
\cventry{2011}{Software Engineering Laboratory}{\textbf{[Java, Maven, Jenkins, git, \LaTeX]}}{Uni Passau}{}{
  \begin{itemize}
    \item Graded 1.0
    \item required course for every Information Technology student
    \item funCKit is an editor and simulator for digital circuits
    \item the source is released under the GPL\\ (\url{https://github.com/mindrunner/funCKit})
  \end{itemize}
}
\cventry{2005--2008}{Theoretical Vocational Training}{Vocational School}{Elmshorn}{IT specialist}{
  \begin{itemize}
    \item Software Developer
  \end{itemize}
}
\cventry{2005--2008}{Practical Vocational Training}{TecBOS GmbH}{Halstenbek}{IT specialist}{
  \begin{itemize}
    \item Software Developer
  \end{itemize}
}
\cventry{1995--2004}{High School}{Dietrich-Bonhoeffer-Gymnasium}{Quickborn}{}{}

\section{projects}
\cventry{2015}{Android Application}{\textbf{[Java, Android, Gradle, git-flow, Jira]}}{}{}{
  \begin{itemize}
    \item development of an android app for a social network startup company
    \item focus on cutting edge android technology
    \item REST client / ContentProvider / SyncProvider / material design
  \end{itemize}
}
\cventry{2014}{TouchIt}{\textbf{[Java, Android, Gradle, git, libGDX]}}{}{}{
  \begin{itemize}
    \item development of an android game based on libGDX
    \item published in Google's play store
    \item published the source code (GPLv3)
    \item \url{https://github.com/mindrunner/touch-it}
  \end{itemize}
}
\cventry{2013--2014}{brightup GmbH}{\textbf{[Java, C, .NET, OpenEmbedded, git-flow, Jira, Jenkins, BuildBot, Android, iOS]}}{}{http://www.brightup.de}{
  \begin{itemize}
    \item development of hardware and software for brightup’s central unit and z-wave plugs
    \item maintenance of general development tools and work flows (bug tracker, version control system, continuous integration)
    \item consulting for cloud, smartphone applications and general technical decisions
  \end{itemize}
}
\cventry{2013}{PhotoBooth and Quickshot}{\textbf{[Java, Redmine, Jenkins, EDSDK (Canon)]}}{}{}{
  \begin{itemize}
    \item two kiosk machine applications for a photo studio
  \end{itemize}
}
\cventry{2013}{hantPlayer}{\textbf{[Objective-C, Redmine, AppKit]}}{}{\url{http://www.hantmade.com}}{
  \begin{itemize}
    \item a native video player with special tagging functions
  \end{itemize}
}
\cventry{2013}{Stomt}{\textbf{[Java, MongoDB, Spring, git-flow]}}{}{\url{http://www.stomt.com}}{
  \begin{itemize}
    \item consulting and the developing of backend and Android-App via pro bono
  \end{itemize}
}
\cventry{2012}{HydrostopEU}{\textbf{[Joomla, PHP]}}{}{\url{http://www.hydrostopeu.com}}{
  \begin{itemize}
    \item complete webpage redesign
  \end{itemize}
}
\cventry{2012}{Nordflair Ticketsystem}{\textbf{[Java, Maven, Jenkins, JPA]}}{}{\url{http://nordflair-events.de/}}{
  \begin{itemize}
    \item co-working on a web-based ticket and entry management system
    \item developing the Java-Frontend entry terminal
  \end{itemize}
}
\cventry{2012}{Templogger}{\textbf{[Java, Maven, Jenkins, JPA]}}{}{}{
  \begin{itemize}
    \item custom service which receives data from GSM-Datalogger: once received, it stores the information into a database and sends warning e-mails according to threshold values
  \end{itemize}
}
\cventry{2010--2011}{usAR-Hamburg}{\textbf{[Java, Maven, Jenkins, JPA, iOS]}}{freifalt GmbH}{\url{http://www.freifalt.com}}{
  \begin{itemize}
    \item iOS-App which shows up public transportation based on augmented reality
  \end{itemize}
}
\cventry{2005--2009}{TecBOS.replication}{\textbf{[Delphi]}}{TecBOS GmbH}{\url{http://tecbos.com}}{
  \begin{itemize}
    \item developing a set of windows services with the aim to automatically replicate multiple TecBOS-Databases
  \end{itemize}
}
\cventry{2005--2009}{TecBOS.mobile}{\textbf{[C, Sqlite]}}{TecBOS GmbH}{\url{http://tecbos.com}}{
  \begin{itemize}
    \item Mobile Windows CE Devices can exchange data with the TecBOS.solutions
  \end{itemize}
}
\cventry{2005--2009}{TecBOS.webaccess}{\textbf{[C\#, firebird]}}{TecBOS GmbH}{\url{http://tecbos.com}}{
  \begin{itemize}
    \item porting the native TecBOS.solutions into a web application
  \end{itemize}
}
\cventry{2005--2009}{TecBOS.solutions}{\textbf{[Delphi, firebird]}}{TecBOS GmbH}{\url{http://tecbos.com}}{
  \begin{itemize}
    \item maintaining and developing parts of the application as part of my apprenticeship and subsequent employment
  \end{itemize}
}
\cventry{2008}{CMS}{\textbf{[MySQL, PHP]}}{}{}{
  \begin{itemize}
    \item developing a whole content management system within one week
  \end{itemize}
}
\cventry{2005--2008}{Web shop}{\textbf{[MySQL, PHP]}}{fontinform GmbH}{\url{http://www.fontinform.com}}{
  \begin{itemize}
    \item planning and development of a web shop for digital fonts
  \end{itemize}
}
\section{tech skills}
\includegraphics[width=0.7\textwidth]{skillcloud}
%\cvline{Basic}{GTK, Perl, Python, VHDL}{}{}
%\cvline{Advanced}{Android, Bash, C, C++, C\#, Delphi, iOS, Maven, MongoDB, MySQL, PHP}{}{}
%\cvline{Expert}{Debian, Gentoo, Git, OpenEmbedded, Java, \LaTeX, Proxmox, Ubuntu, Yocto}{}{}
%\cvline{Tools}{CMake, Eclipse, Intellij IDE, Make, Netbeans, Vim}{}{}
%\cvline{Miscellaneous}{Office, OS X, Windows}{}{}

\section{languages}
\cvline{German}{Native}{}
\cvline{English}{Fluent}{}
\cvline{French}{Basic}{}
\cvline{Polish}{Beginner}{}

\section{personal interests}
\cvline{}{\begin{itemize}
    \item rock climbing
    \item trail running
    \item meditation
    \item yoga
    \item mountain biking
    \item travelling
\end{itemize}}
~\\
~\\
~\\
\today\\ % Aktuelles Datum und Stadt
\end{document}

